\section{1A: Tallfølger}

\blueheader
\begin{frame}
\frametitle{1A: Tallfølger}
\begin{blue*}{Tallfølger}
Tilfeldig tallfølge
\begin{equation*}
\{3,\,-1,\,7,\,7,\,2\}
\end{equation*}

Endelig tallfølge av partall
\begin{equation*}
\{2,\,4,\,6,\,8,\,10,\,12\}
\end{equation*}

Uendelig følge av oddetall
\begin{equation*}
\{1,\,3,\,5,\,7,\,9,\,\ldots\}
\end{equation*}
\end{blue*}
\end{frame}

\blueheader
\begin{frame}
\frametitle{1A: Formler for tallfølger}
\begin{blue*}{Formler for partall og oddetall}
\begin{equation*}
a_n = 2n, \qquad n=1,2,3,\ldots
\end{equation*}

\medskip
\begin{equation*}
b_n = 2n-1, \qquad n=1,2,3,\ldots
\end{equation*}

Eksempler:
\begin{equation*}
\{a_n\} = \{2,\,4,\,6,\,8,\,10,\,\ldots\}
\end{equation*}

\medskip
\begin{equation*}
\{b_n\} = \{1,\,3,\,5,\,7,\,9,\,\ldots\}
\end{equation*}
\end{blue*}
\end{frame}

\redheader
\begin{frame}
\frametitle{1A: To måter å skrive en tallfølge på}

Tallfølger kan beskrives på to måter:
\begin{itemize}
    \item Ved å \textbf{skrive opp leddene} i rekkefølge, f.eks. 
    \[
    \{2,\,4,\,6,\,8,\,10,\,\ldots\}
    \]
    \item Ved å \textbf{angi en formel} for det $n$-te leddet, f.eks.
    \[
    a_n = 2n
    \]
\end{itemize}

\medskip
Begge metodene beskriver samme tallfølge, men en formel gir oss en 
\textbf{generell oppskrift} slik at vi kan finne hvilket som helst ledd uten å måtte skrive opp alle de foregående.

\end{frame}

\blueheader
\begin{frame}
\frametitle{1A: To typer formler for tallfølger}
\begin{red*}{Eksplisitt formel}
En eksplisitt formel gir en direkte oppskrift på det $n$-te leddet:
\[
a_n = 2n \quad \Rightarrow \quad \{2,\,4,\,6,\,8,\,\ldots\}
\]
Her kan vi finne $a_{100}$ rett fra formelen: $a_{100}=200$.
\end{red*}

\begin{red*}{Rekursiv formel}
En rekursiv formel beskriver hvert ledd ut fra det forrige:
\[
a_1 = 2, \qquad a_{n+1} = a_n + 2
\]
Her må vi starte med første ledd, og bygge videre trinn for trinn.
\end{red*}
\end{frame}

\blueheader
\begin{frame}
\frametitle{1A: Fibonacci-følgen}

\begin{blue*}{Definisjon}
Fibonacci-følgen er definert rekursivt ved
\begin{equation*}
F_1 = 1, \quad F_2 = 1, \quad F_{n+2} = F_{n+1} + F_n \quad \text{for } n \geq 1
\end{equation*}
\end{blue*}

\begin{green*}{Eksempel}
De første leddene blir
\begin{equation*}
\{1,\,1,\,2,\,3,\,5,\,8,\,13,\,21,\,\ldots\}
\end{equation*}

Her ser vi at hvert ledd er summen av de to foregående.
\end{green*}

\end{frame}

\blueheader
\begin{frame}[fragile]
\frametitle{1A: Fibonaccitallene}

\begin{green*}{Eksempel i Python}
\begin{minted}[fontsize=\large]{python}
# Neste tall i følgen er summen av de to foregående tallene
# Husk at indeksene i en liste starter på 0

fib_liste = [1, 1]

for n in range(0, 5):
    a_n = fib_liste[n+1] + fib_liste[n]
    fib_liste.append(a_n)

print(fib_liste)
# Output: [1, 1, 2, 3, 5, 8, 13]
\end{minted}
\end{green*}
\end{frame}


\blueheader
\begin{frame}
\frametitle{1A: Rekker}
\begin{blue*}{Definisjon}
En \textbf{rekke} er summen av leddene i en tallfølge.  
Hvis vi har en tallfølge
\[
\{a_1, a_2, a_3, \ldots\}
\]
så danner vi en rekke ved å summere leddene:
\[
S = a_1 + a_2 + a_3 + \ldots
\]
\end{blue*}
\end{frame}

\begin{frame}{1A: Rekker}
\begin{green*}{Eksempel}
En endelig rekke av partall:
\begin{equation*}
2 + 4 + 6 + 8 = 20
\end{equation*}
Her er summen $20$ den verdien rekken får.
\end{green*}
\end{frame}


\blueheader
\begin{frame}
\frametitle{1A: Summasjonsnotasjon}
\begin{blue*}{Definisjon}
Summasjonsnotasjon bruker symbolet $\sum$ for å skrive summen av mange ledd på en kompakt form:
\begin{equation*}
\sum_{i=1}^{n} a_i = a_1 + a_2 + \ldots + a_n
\end{equation*}
\end{blue*}

\begin{green*}{Eksempel}
\begin{equation*}
\sum_{i=1}^{4} 3i = 3 + 6 + 9 + 12 = 30
\end{equation*}
\end{green*}

\end{frame}
\blueheader
\begin{frame}
\frametitle{1A: Viktig med parenteser (I)}

\begin{green*}{Eksempel med parenteser}
\begin{align*}
\sum_{i=1}^{4} (2i+3) 
&= (2\cdot1+3) + (2\cdot2+3) + (2\cdot3+3) + (2\cdot4+3) \\
&= 5 + 7 + 9 + 11 \\
&= 32
\end{align*}
\end{green*}

\end{frame}

\blueheader
\begin{frame}
\frametitle{1A: Viktig med parenteser (II)}

\begin{green*}{Eksempel uten parenteser}
\begin{align*}
\sum_{i=1}^{4} 2i+3 
&= (2+4+6+8) + 3 \\
&= 20 + 3 \\
&= 23
\end{align*}

\medskip
\textbf{Merk:} Parentesene avgjør hva som er med i selve summeringen.
\end{green*}

\end{frame}
\begin{frame}
\frametitle{1A: Summen av de hundre første heltall}

\begin{red*}{Viktig eksempel}
Vi ønsker å regne ut summen
\begin{equation*}
1 + 2 + 3 + 4 + \ldots + 100
\end{equation*}

Formelen for summen av de hundre første naturlige tallene er

\begin{align*}
\sum_{i=1}^{100} i 
&= 1 + 2 + 3 + 4 + \ldots + 100 \\
&= \frac{100 \cdot 101}{2} \\
&= 5050
\end{align*}
\end{red*}
\end{frame}

\greenheader
\begin{frame}[fragile]
\frametitle{1A: Tallfølger og rekker i Python} 
\begin{minted}[fontsize=\normalsize]{python}

# Eksempel med for-løkke
minListe = []

for n in range(1,6):
  minListe.append(n)
  
print(minListe) # Output:  [1, 2, 3, 4, 5]

# Samme eksempel med List Comprehension (kommer ikke på prøver)
minListe = [n for n in range(1, 6)]   

print(minListe) # Output:  [1, 2, 3, 4, 5]
\end{minted}
\end{frame}

\greenheader
\begin{frame}[fragile]
\frametitle{1A: Tallfølger og rekker i Python}
% Eksempel 2: 10 første partall
\begin{minted}[fontsize=\small]{python}
# De 10 første partallene og summen

# Eksempel med for-løkke
minListe = []

for n in range(1,10+1):
  minListe.append(2*n)
  
print(minListe)      # Output:  [2, 4, 6, 8, 10, 12, 14, 16, 18, 20]
print(sum(minListe)) # Output: 110

# Samme eksempel med List Comprehension (kommer ikke på prøver)
minListe = [2*n for n in range(1, 10+1)]   

print(minListe)      # Output:  [2, 4, 6, 8, 10, 12, 14, 16, 18, 20]
print(sum(minListe)) # Output: 110

\end{minted}
\end{frame}





