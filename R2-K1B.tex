\section{1B: Bevis}

\blueheader
\begin{frame}
\frametitle{1B: Implikasjonspilen}

\begin{blue*}{Definisjon}
Implikasjonspilen 
\begin{center}
$\Rightarrow$ 
\end{center}
brukes i logikk og matematikk for å uttrykke at en påstand medfører en annen.

\medskip
Hvis $P$ er en påstand og $Q$ er en påstand, så betyr
\[
P \Rightarrow Q
\]
at \textit{hvis $P$ er sann, så er også $Q$ sann}.

\end{blue*}

\end{frame}

\setbeamercolor{palette primary}{bg=mygreen, fg=white}

\blueheader
\begin{frame}
\frametitle{1B: Eksempel}

Vi har to påstander.

\medskip
\begin{itemize}
    \item[] $P$: «Du er i Harstad» \\
    \item[] $Q$: «Du er i Troms fylke»
\end{itemize}


\medskip
Da kan vi skrive
\[
P \Rightarrow Q
\]

Dette er en sann implikasjon, fordi det å være i Harstad innebærer å være i Troms fylke.


\begin{red*}{Det motsatte er ikke nødvendigvis sant}
  
$P \Rightarrow Q$ betyr ikke nødvendigvis at $Q \Rightarrow P$.
\end{red*}

\end{frame}


\begin{frame}
\frametitle{1B: Direkte bevis}


La $P$: «$n$ er et partall»  
og $Q$: «$n^2$ er et partall».

\medskip
Vi skal bevise implikasjonen
\[
P \Rightarrow Q
\]

\medskip
Anta at $P$ er sann, dvs. at $n$ er et partall.  
Da kan vi skrive
\[
n = 2k \quad \text{for et heltall } k.
\]

Da blir
\[
n^2 = (2k)^2 = 4k^2 = 2\,(2k^2).
\]

\medskip
Siden $n^2$ kan skrives som $2 \cdot$ (et heltall), følger det at $Q$ er sann.  
\hfill$\blacksquare$


\end{frame}


\blueheader
\begin{frame}
\frametitle{1B: Kontrapositivt}

\begin{blue*}{Definisjon}
Hvis  implikasjonen
\[
P \Rightarrow Q
\]
er sann,  
da er også
\[
\lnot Q \Rightarrow \lnot P
\]
sann.

\medskip
Dette kaller vi det \textbf{kontraposive} av en påstand.
\end{blue*}

\end{frame}

\greenheader
\begin{frame}
\frametitle{1B: Eksempel (I) }


La $P$: «Du er i Harstad»  
og $Q$: «Du er i Troms fylke».

\medskip
Vi har implikasjonen
\[
P \Rightarrow Q
\]

Det kontrapositive blir
\[
\lnot Q \Rightarrow \lnot P
\]

\medskip
Altså: «Hvis du \emph{ikke} er i Troms fylke, så er du \emph{ikke} i Harstad.»  
Dette er en sann påstand.


\end{frame}
\greenheader
\begin{frame}
\frametitle{1B: Eksempel (II)}
Vi har to påstander, P og Q.
\begin{itemize}
    \item[] $P$: $n^2$ er partall\\
    \item[] $Q$: $n$ er partall.
\end{itemize}   

\medskip
Vi skal vise $P \Rightarrow Q$ ved å bevise kontrapositive: $\lnot Q \Rightarrow \lnot P$.

\medskip
Anta $\lnot Q$, dvs. $n$ er oddetall.  Da er $n=2k+1$ for et heltall $k$.  
\[
n^2 = (2k+1)^2 = 4k^2 + 4k + 1 = 2(2k^2+2k)+1,
\]
som er oddetall, altså $\lnot P$.

\medskip
Dermed har vi vist $\lnot Q \Rightarrow \lnot P$, og dermed $P \Rightarrow Q$. \hfill $\blacksquare$


\end{frame}

\blueheader
\begin{frame}{1B: Induksjonsbevis}

\begin{blue*}{Oppskrift på induksjon}
For å bevise en påstand $P(n)$ for alle $n \in \mathbb{N}$:

\begin{enumerate}
    \item \textbf{Induksjonsbasis:} Vis at påstanden stemmer for $n=1$.
    \item \textbf{Induksjonshypotese:} Anta at påstanden stemmer for $n=k$.
    \item \textbf{Induksjonstrinn:} Bruk antakelsen til å vise at påstanden
          også stemmer for $n=k+1$.
    \item \textbf{Konklusjon:} Dermed gjelder påstanden for alle $n \in \mathbb{N}$.
\end{enumerate}

\end{blue*}

\end{frame}


\greenheader
\begin{frame}
\frametitle{1B: Induksjonsbevis (I)}

\textbf{Påstand:}

For alle $n \in \mathbb{N}$ gjelder
\[
1 + 2 + 3 + \dots + n = \frac{n(n+1)}{2}.
\]


\textbf{Induksjonsbasis:}

Vi viser at påstanden er sann for $n=1$:
\[
1 = \frac{1 \cdot (1+1)}{2} = 1.
\]
Dermed er basistilfellet oppfylt.


\end{frame}


\greenheader
\begin{frame}
\frametitle{1B: Induksjonsbevis (II)}

Anta at påstanden gjelder for $n=k$, altså
\[
1 + 2 + 3 + \dots + k = \frac{k(k+1)}{2}.
\]

Vi viser at den da gjelder for $n=k+1$:
\begin{align*}
1 + 2 + 3 + \dots + k + (k+1)
&= \frac{k(k+1)}{2} + (k+1) \\
&= \frac{k(k+1) + 2(k+1)}{2} \\
&= \frac{(k+1)(k+2)}{2}.
\end{align*}

Dette er akkurat formelen med $n=k+1$.  
Dermed gjelder påstanden for alle $n \in \mathbb{N}$. \hfill $\blacksquare$


\end{frame}



%-----------------
\greenheader
\begin{frame}
\frametitle{1B: Induksjonsbevis for en rekursiv sammenheng (I)}
Følgen $\{a_n\}$ er gitt ved
\[
a_1=-3, \qquad  a_{n+1}=a_n+2n-3.
\]

Vis ved induksjon at
\[
a_n = n(n-4), \quad n\in\mathbb{N}.
\]

\medskip
\textbf{Påstand:} 

For alle $n\in\mathbb{N}$ gjelder $a_n = n(n-4)$.

\medskip
\textbf{Induksjonsbasis:}


Starter med å sjekke om formelen stemmer for basistilfellet $n=1$:
\[
a_1=-3, \qquad 1(1-4)=-3.
\]


\end{frame}


\greenheader
\begin{frame}
\frametitle{1B: Induksjonsbevis for en rekursiv sammenheng  (II)}


Antar at $a_n = n(n-4)$  gir oss det n'te leddet i følgen 

\[
a_1=-3 \quad \wedge \quad  a_{n+1}=a_n+2n-3
\]
Dette gir:
\begin{align*}
a_{k+1} &= a_k + 2k - 3 \\
        &= k(k-4) + 2k - 3 \\
        &= k^2 - 2k - 3 \\
        &= (k+1)(k-3) \\
        &= (k+1)\bigl((k+1)-4\bigr).
\end{align*}

Dette er formelen for det n'te leddet i følgen for $n=k+1$.

\hfill $\blacksquare$


\end{frame}
